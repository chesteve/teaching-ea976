%%%%%%%%%%%%%%%%%%%%%%%%%%%%%%%%%%%%%%%%%%%%%%%%%%%%%%%%%%%%%%%%%%%%%%
% LaTeX Example: Project Report
%
% Source: http://www.howtotex.com
%
% Feel free to distribute this example, but please keep the referral
% to howtotex.com
% Date: March 2011 
% 
%%%%%%%%%%%%%%%%%%%%%%%%%%%%%%%%%%%%%%%%%%%%%%%%%%%%%%%%%%%%%%%%%%%%%%
% How to use writeLaTeX: 
%
% You edit the source code here on the left, and the preview on the
% right shows you the result within a few seconds.
%
% Bookmark this page and share the URL with your co-authors. They can
% edit at the same time!
%
% You can upload figures, bibliographies, custom classes and
% styles using the files menu.
%
% If you're new to LaTeX, the wikibook is a great place to start:
% http://en.wikibooks.org/wiki/LaTeX
%
%%%%%%%%%%%%%%%%%%%%%%%%%%%%%%%%%%%%%%%%%%%%%%%%%%%%%%%%%%%%%%%%%%%%%%
% Edit the title below to update the display in My Documents
%\title{Relatório Atividade Complementar FLOSS EA976}
%
%%% Preamble

\documentclass[12pt,a4paper]{article} % Use A4 paper with a 12pt font size - different paper sizes will require manual recalculation of page margins and border positions

\usepackage{marginnote} % Required for margin notes
\usepackage{wallpaper} % Required to set each page to have a background
\usepackage{lastpage} % Required to print the total number of pages
\usepackage[left=1.3cm,right=2.0cm,top=1.8cm,bottom=5.0cm,marginparwidth=3.4cm]{geometry} % Adjust page margins
\usepackage{amsmath} % Required for equation customization
\usepackage{amssymb} % Required to include mathematical symbols
\usepackage{xcolor} % Required to specify colors by name

\usepackage{fancyhdr} % Required to customize headers
\usepackage[brazil]{babel}
\usepackage[latin1]{inputenc}
%\usepackage[T1]{fontenc}
\usepackage{graphicx}
\usepackage{pstricks}
\usepackage{subfigure}
\usepackage{caption}  % legendas nas figuras
\captionsetup{justification=centering,labelfont=bf}
\usepackage{textcomp}
\setlength{\headheight}{80pt} % Increase the size of the header to accommodate meta-information
\pagestyle{fancy}\fancyhf{} % Use the custom header specified below
\renewcommand{\headrulewidth}{0pt} % Remove the default horizontal rule under the header

\setlength{\parindent}{0cm} % Remove paragraph indentation
\newcommand{\tab}{\hspace*{2em}} % Defines a new command for some horizontal space

\newcommand\BackgroundStructure{ % Command to specify the background of each page
\setlength{\unitlength}{1mm} % Set the unit length to millimeters

\setlength\fboxsep{0mm} % Adjusts the distance between the frameboxes and the borderlines
\setlength\fboxrule{0.5mm} % Increase the thickness of the border line
\put(10, 20pr){\fcolorbox{black}{gray!5}{\framebox(155,247){}}} % Main content box
\put(165, 20){\fcolorbox{black}{gray!10}{\framebox(37,247){}}} % Margin box
\put(10, 262){\fcolorbox{black}{white!10}{\framebox(192, 25){}}} % Header box
\put(175, 263){\includegraphics[height=23mm,keepaspectratio]{}} % Logo box - maximum height/width: 
}

%----------------------------------------------------------------------------------------
%	HEADER INFORMATION
%----------------------------------------------------------------------------------------

\fancyhead[L]{\begin{tabular}{l r | l r} % The header is a table with 4 columns
\textbf{2S/2014} & EA-976 & \textbf{P\'agina:} & \thepage/\pageref{LastPage} \\ % Project name and page count
\textbf{Atividade:} & Complementar \#4 & \textbf{Data:} & 27/10/14 \\ % Job number and last updated date
\textbf{Professor:} & Christian E. Rothenberg & \textbf{Assunto:} & Caracterização  projeto \textit{open source}  \\ % Version and reviewed date
\textbf{Projeto:} & Nome projeto open source & \textbf{Autor:} & Erik Perillo (RA: 135582) \\ % Designer and reviewer
\end{tabular}}

%----------------------------------------------------------------------------------------

\begin{document}

%\AddToShipoutPicture{\BackgroundStructure} % Set the background of each page to that specified above in the header information section

%----------------------------------------------------------------------------------------
%	DOCUMENT CONTENT
%----------------------------------------------------------------------------------------

\section{Relat\'orio t\'ecnico descritivo sobre projeto de c\'odigo livre} 

Este modelo reduzido visa proporcionar um roteiro pr\'atico para a reda\,c\~ao do relatório da atvidade complementar \#4, na forma de relat\'orio simplificado. 

Esta se\,c\~ao \'e destinada a apresentar os objetivos do trabalho.\\

\subsection{Descrição do projeto}

Node.js é uma plataforma para se construir aplicações de rede rápidas e escaláveis. Feita sobre o motor JavaScript do Google Chrome (V8), com um modelo de I/O orientado a eventos, a plataforma foi pensada para aplicações server-side com grande troca de dados com um grande número de clientes.

\section{Caracterização do projeto de código livre} 
Pesquise sobre o projeto para responder as seguintes questões.


\subsection{Desenvolvimento}


\begin{itemize}
\item Existe um local dedicado para o desenvolvimento? - Não existe um local físico, mas sim um repositório git (github.com/joyent/node) onde todo o desenvolvimento acontece e o projeto é atualizado.
\item É possível extrair o atual código fonte a partir de um repositório público de código fonte? - Sim, como dito no item anterior, todo o código fonte está explicitamente disponível no repositório do github do projeto.
\item Quão grande é o código?
\item Quais são as principais linguagens de programação? - Node.js é feito basicamente em C/C++.
\item A utilização do pacote depende de algum outro software proprietário ou de código fonte aberto? - Node para Unix tem poucas dependências, sendo listadas como: GCC, G+++, Python (2.6 ou 2.7), GNU Make, libexecinfo. Todos são softwares de código aberto. Entretanto, para ambientes Windows, precisa de Visual Studio para ser construído.
\item É possível calcular o número de \textit{downloads} ou usuários de uma versão em particular? - O github permite visualizar isto pela seção "Traffic graphs", entretanto, esta não está visível no repositório do node.
\end{itemize}

\subsection{Licença Software Livre}


\begin{itemize}
\item Quem são os patrocinadores que contribuem para a sustentabilidade do projeto? - A empresa por trás do node.js e que hoje o mantém é a Joyent. Para a manutenção e evolução do projeto em termos de desenvolvimento, há diversos contribuidores ao redor do mundo. Apesar de ter pessoas oficializadas como programadores-cabeça do projeto, qualquer um é encorajado a se engajar na contribuição.
\item Quem detém os direitos autorais do código? - Sua empresa, Joyent Inc.
\item O projeto está sob qual tipo de licença de código aberto? - Ele é lançado sob a "MIT license".
\item Por que os responsáveis pelo projeto escolheram a licença de código aberto? - Pela capacidade do projeto de ser dinamicamente mantido e evoluído por várias e qualquer pessoa ao redor do mundo com interesse em tal. Node nasceu com isso em mente.
\end{itemize}

\subsection{Governança}


\begin{itemize}
\item Existem quantos desenvolvedores alocados para o projeto?
\item Quantos \textit{committers}, também conhecidos por desenvolvedores que podem realizar mudanças propostas, o projeto possui?
\item O que você pode dizer sobre o modelo de governança de código fonte aberto?
\end{itemize}

\subsection{Manutenção}


\begin{itemize}
\item Gerenciamento de \textit{releases}: Qual o número e frequência de \textit{releases}? - O github mostra um total de 244 releases. A frequência exata não pôde ser calculada mas, considerando-se que node foi lançado em meados de 2009, pode-se dizer que é alta.
\item Comunicação: Existe um canal de comunicação útil e ativo para a comunidade / suporte ao usuário? - Sim, vários. No site do node.js (seção "Community"), constam: lista de emails, canais de IRC, um github issues list, assim como alguns blogs, a maioria agregada no "Planet Node".
\item Existe um \textit{bugtracker} (rastreamento de bugs) com uma lista de bugs corrigidos/pendentes de correção? - Sim, presente em https://github.com/joyent/node/issues
\item Existe um plano de metas para planos futuros? Existe evidência que o plano de metas foi utilizado no passado? - Não foram encontrados planos oficiais, apenas aqueles discutidos em conferências sobre o node.js, mailing lists e discussão do github. Toda mudança no projeto deve ser pesadamente discutida antes, as mais críticas sendo encorajadas no site a serem discutidas no github issues list.
\item Existe consultoria comercial, treinamento ou consulta disponível para o projeto? A partir de múltiplos prestadores de serviços?  - Sim, a partir de vários prestadores, um deles sendo o "StrongLoop"(.com).       
\end{itemize}
        

\par\vspace{\baselineskip}

%----------------------------------------------------------------------------------------

\end{document}
