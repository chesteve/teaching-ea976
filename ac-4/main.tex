%%%%%%%%%%%%%%%%%%%%%%%%%%%%%%%%%%%%%%%%%%%%%%%%%%%%%%%%%%%%%%%%%%%%%%
% LaTeX Example: Project Report
%
% Source: http://www.howtotex.com
%
% Feel free to distribute this example, but please keep the referral
% to howtotex.com
% Date: March 2011 
% 
%%%%%%%%%%%%%%%%%%%%%%%%%%%%%%%%%%%%%%%%%%%%%%%%%%%%%%%%%%%%%%%%%%%%%%
% How to use writeLaTeX: 
%
% You edit the source code here on the left, and the preview on the
% right shows you the result within a few seconds.
%
% Bookmark this page and share the URL with your co-authors. They can
% edit at the same time!
%
% You can upload figures, bibliographies, custom classes and
% styles using the files menu.
%
% If you're new to LaTeX, the wikibook is a great place to start:
% http://en.wikibooks.org/wiki/LaTeX
%
%%%%%%%%%%%%%%%%%%%%%%%%%%%%%%%%%%%%%%%%%%%%%%%%%%%%%%%%%%%%%%%%%%%%%%
% Edit the title below to update the display in My Documents
%\title{Relatório Atividade Complementar FLOSS EA976}
%
%%% Preamble

\documentclass[12pt,a4paper]{article} % Use A4 paper with a 12pt font size - different paper sizes will require manual recalculation of page margins and border positions

\usepackage{marginnote} % Required for margin notes
\usepackage{wallpaper} % Required to set each page to have a background
\usepackage{lastpage} % Required to print the total number of pages
\usepackage[left=1.3cm,right=2.0cm,top=1.8cm,bottom=5.0cm,marginparwidth=3.4cm]{geometry} % Adjust page margins
\usepackage{amsmath} % Required for equation customization
\usepackage{amssymb} % Required to include mathematical symbols
\usepackage{xcolor} % Required to specify colors by name

\usepackage{fancyhdr} % Required to customize headers
\usepackage[brazil]{babel}
\usepackage[utf8]{inputenc}
%\usepackage[T1]{fontenc}
\usepackage{graphicx}
\usepackage{pstricks}
\usepackage{subfigure}
\usepackage{caption}  % legendas nas figuras
\captionsetup{justification=centering,labelfont=bf}
\usepackage{textcomp}
\setlength{\headheight}{80pt} % Increase the size of the header to accommodate meta-information
\pagestyle{fancy}\fancyhf{} % Use the custom header specified below
\renewcommand{\headrulewidth}{0pt} % Remove the default horizontal rule under the header

\setlength{\parindent}{0cm} % Remove paragraph indentation
\newcommand{\tab}{\hspace*{2em}} % Defines a new command for some horizontal space

\newcommand\BackgroundStructure{ % Command to specify the background of each page
\setlength{\unitlength}{1mm} % Set the unit length to millimeters

\setlength\fboxsep{0mm} % Adjusts the distance between the frameboxes and the borderlines
\setlength\fboxrule{0.5mm} % Increase the thickness of the border line
\put(10, 20pr){\fcolorbox{black}{gray!5}{\framebox(155,247){}}} % Main content box
\put(165, 20){\fcolorbox{black}{gray!10}{\framebox(37,247){}}} % Margin box
\put(10, 262){\fcolorbox{black}{white!10}{\framebox(192, 25){}}} % Header box
\put(175, 263){\includegraphics[height=23mm,keepaspectratio]{}} % Logo box - maximum height/width: 
}

%----------------------------------------------------------------------------------------
%	HEADER INFORMATION
%----------------------------------------------------------------------------------------

\fancyhead[L]{\begin{tabular}{l r | l r} % The header is a table with 4 columns
\textbf{2S/2014} & EA-976 & \textbf{P\'agina:} & \thepage/\pageref{LastPage} \\ % Project name and page count
\textbf{Atividade:} & Complementar \#4 & \textbf{Data:} & 27/10/14 \\ % Job number and last updated date
\textbf{Professor:} & Christian E. Rothenberg & \textbf{Assunto:} & Caracterização  projeto \textit{open source}  \\ % Version and reviewed date
\textbf{Projeto:} & Nome projeto open source & \textbf{Autor:} & Nome Aluno (RA: XXXXXX) \\ % Designer and reviewer
\end{tabular}}

%----------------------------------------------------------------------------------------

\begin{document}

%\AddToShipoutPicture{\BackgroundStructure} % Set the background of each page to that specified above in the header information section

%----------------------------------------------------------------------------------------
%	DOCUMENT CONTENT
%----------------------------------------------------------------------------------------

\section{Relat\'orio t\'ecnico descritivo sobre projeto de c\'odigo livre} 

\subsection{Descrição do projeto}
1.1	O openSUSE Linux é um sistema operacional código aberto baseado no kernel do Linux para PC. Ele é utilizado por consumidores, pequenas empresas e desenvolvedores. O sistema é desenvolvido pela comunidade openSUSE de forma gratuita.



\section{Caracterização do projeto de código livre} 


\subsection{Desenvolvimento}


\begin{itemize}
\item Existe um local dedicado para o desenvolvimento?
Não há um local dedicado para o desenvolvimento, uma vez que ele é feito pela comunidade por meio de um portal web.
\item É possível extrair o atual código fonte a partir de um repositório público de código fonte?
É possível extrair o código fonte atual a partir de um repositório público.
\item Quão grande é o código?
Uma estimativa para uma distribuição de Linux é que o código tenha por volta de 200 milhões de linhas.
\item Quais são as principais linguagens de programação?
PHP, Go, Perl, Python e Ruby são as principais linguagens de programação usadas no openSUSE.
\item A utilização do pacote depende de algum outro software proprietário ou de código fonte aberto?
O openSUSE oferece aos usuários um repositório de software não livre, contém blobs binários de drivers, mas usa o kernel de código aberto do Linux.
\item É possível calcular o número de \textit{downloads} ou usuários de uma versão em particular?
Sim, é possível calcular o número de downloads ou usuários de uma versão particular. Tais dados podem ser obtidos nessa página: https://lizards.opensuse.org/2013/08/23/more-on-statistics/.
\end{itemize}

\subsection{Licença Software Livre}


\begin{itemize}
\item Quem são os patrocinadores que contribuem para a sustentabilidade do projeto?
A empresa Novell é a patrocinadora que contribui para a sustentabilidade do projeto.
\item Quem detém os direitos autorais do código?
Os direitos autorais são de propriedade do The openSUSE Project.
\item O projeto está sob qual tipo de licença de código aberto?
O projeto está sob a Licença Pública Geral GNU.
\item Por que os responsáveis pelo projeto escolheram a licença de código aberto?
Os responsáveis escolheram a licença de código aberto para que toda a comunidade contribuísse e desenvolvesse o sistema.
\end{itemize}

\subsection{Governança}


\begin{itemize}
\item Existem quantos desenvolvedores alocados para o projeto?
Foram 296 desenvolvedores que contribuíram para o projeto.
\item Quantos \textit{committers}, também conhecidos por desenvolvedores que podem realizar mudanças propostas, o projeto possui?
O desenvolvimento é aberto à comunidade, não há um número específico de committers.
\item O que você pode dizer sobre o modelo de governança de código fonte aberto?
 modelo de governança de código fonte aberto é muito adequado para o openSUSE, uma vez que um projeto aberto para o desenvolvimento de um sistema operacional pela comunidade pode levar a um sistema bem estável e amplo, atendendo diversas necessidades.
\end{itemize}

\subsection{Manutenção}


\begin{itemize}
\item Gerenciamento de \textit{releases}: Qual o número e frequência de \textit{releases}?
Foram 43 releases, com uma frequência aproximada de 3 releases por ano.
\item Comunicação: Existe um canal de comunicação útil e ativo para a comunidade / suporte ao usuário?
O canal de comunicação ativo é um portal web do openSUSE.
\item Existe um \textit{bugtracker} (rastreamento de bugs) com uma lista de bugs corrigidos/pendentes de correção?
Sim, o bugtracker utilizado é o Bugzilla.
\item Existe um plano de metas para planos futuros? Existe evidência que o plano de metas foi utilizado no passado?
Sim, existe um plano de metas para planos futuros, como atualizações e implementações a serem feitas. Como há várias versões do plano de metas, ele certamente é eventualmente atualizado e assim, foi usado no passado.
\item Existe consultoria comercial, treinamento ou consulta disponível para o projeto? A partir de múltiplos prestadores de serviços?
Sim, existe treinamento para openSUSE, inclusive com certificação.
\end{itemize}
        

\par\vspace{\baselineskip}

%----------------------------------------------------------------------------------------

\end{document}
