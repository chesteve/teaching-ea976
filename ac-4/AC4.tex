%%%%%%%%%%%%%%%%%%%%%%%%%%%%%%%%%%%%%%%%%%%%%%%%%%%%%%%%%%%%%%%%%%%%%%
% LaTeX Example: Project Report
%
% Source: http://www.howtotex.com
%
% Feel free to distribute this example, but please keep the referral
% to howtotex.com
% Date: March 2011 
% 
%%%%%%%%%%%%%%%%%%%%%%%%%%%%%%%%%%%%%%%%%%%%%%%%%%%%%%%%%%%%%%%%%%%%%%
% How to use writeLaTeX: 
%
% You edit the source code here on the left, and the preview on the
% right shows you the result within a few seconds.
%
% Bookmark this page and share the URL with your co-authors. They can
% edit at the same time!
%
% You can upload figures, bibliographies, custom classes and
% styles using the files menu.
%
% If you're new to LaTeX, the wikibook is a great place to start:
% http://en.wikibooks.org/wiki/LaTeX
%
%%%%%%%%%%%%%%%%%%%%%%%%%%%%%%%%%%%%%%%%%%%%%%%%%%%%%%%%%%%%%%%%%%%%%%
% Edit the title below to update the display in My Documents
%\title{Relatório Atividade Complementar FLOSS EA976}
%
%%% Preamble

\documentclass[12pt,a4paper]{article} % Use A4 paper with a 12pt font size - different paper sizes will require manual recalculation of page margins and border positions

\usepackage{marginnote} % Required for margin notes
\usepackage{wallpaper} % Required to set each page to have a background
\usepackage{lastpage} % Required to print the total number of pages
\usepackage[left=1.3cm,right=2.0cm,top=1.8cm,bottom=5.0cm,marginparwidth=3.4cm]{geometry} % Adjust page margins
\usepackage{amsmath} % Required for equation customization
\usepackage{amssymb} % Required to include mathematical symbols
\usepackage{xcolor} % Required to specify colors by name

\usepackage{fancyhdr} % Required to customize headers
\usepackage[brazil]{babel}
\usepackage[utf8]{inputenc}
\usepackage[latin1]{inputenc}
%\usepackage[T1]{fontenc}
\usepackage{graphicx}
\usepackage{pstricks}
\usepackage{subfigure}
\usepackage{caption}  % legendas nas figuras
\captionsetup{justification=centering,labelfont=bf}
\usepackage{textcomp}
\setlength{\headheight}{80pt} % Increase the size of the header to accommodate meta-information
\pagestyle{fancy}\fancyhf{} % Use the custom header specified below
\renewcommand{\headrulewidth}{0pt} % Remove the default horizontal rule under the header

\setlength{\parindent}{0cm} % Remove paragraph indentation
\newcommand{\tab}{\hspace*{2em}} % Defines a new command for some horizontal space

\newcommand\BackgroundStructure{ % Command to specify the background of each page
\setlength{\unitlength}{1mm} % Set the unit length to millimeters

\setlength\fboxsep{0mm} % Adjusts the distance between the frameboxes and the borderlines
\setlength\fboxrule{0.5mm} % Increase the thickness of the border line
\put(10, 20pr){\fcolorbox{black}{gray!5}{\framebox(155,247){}}} % Main content box
\put(165, 20){\fcolorbox{black}{gray!10}{\framebox(37,247){}}} % Margin box
\put(10, 262){\fcolorbox{black}{white!10}{\framebox(192, 25){}}} % Header box
\put(175, 263){\includegraphics[height=23mm,keepaspectratio]{}} % Logo box - maximum height/width: 
}

%----------------------------------------------------------------------------------------
%	HEADER INFORMATION
%----------------------------------------------------------------------------------------

\fancyhead[L]{\begin{tabular}{l r | l r} % The header is a table with 4 columns
\textbf{2S/2014} & EA976 & \textbf{P\'agina:} & \thepage/\pageref{LastPage} \\ % Project name and page count
\textbf{Atividade:} & Complementar \#4 & \textbf{Data:} & 27/10/14 \\ % Job number and last updated date
\textbf{Professor:} & Christian E. Rothenberg & \textbf{Assunto:} & Caracterização  projeto \textit{open source}  \\ % Version and reviewed date
\textbf{Projeto:} & Apache Hadoop & \textbf{Autor:} & João Henrique S. Hoffmam (R.A.: 136249) \\ % Designer and reviewer
\end{tabular}}

%----------------------------------------------------------------------------------------

\begin{document}

%\AddToShipoutPicture{\BackgroundStructure} % Set the background of each page to that specified above in the header information section

%----------------------------------------------------------------------------------------
%	DOCUMENT CONTENT
%----------------------------------------------------------------------------------------

\section{Relatório técnico descritivo sobre projeto de código livre} 

Este modelo reduzido visa proporcionar um roteiro prático para a redação do relatório da atvidade complementar \#4, na forma de relatório simplificado. 

Esta seção é destinada a apresentar os objetivos do trabalho.\\

\subsection{Descrição do projeto}

\emph{Apache Hadoop} é um projeto de \emph{software open-source} que permite o processamento distribuído de grandes quantidades de dados em \emph{clusters}. Em outras palavras, é um \emph{framework} que trabalha o conceito de \emph{Big Data}, dividindo os dados em blocos que são distribuídos pelos nós dentro do \emph{cluster}, para serem processados paralelamente por meio de um algoritmo de \emph{Map-Reduce}. Como principal vantagem, o sistema apresenta grande capacidade de tolerar falhas, tornando-o uma abordagem muito mais barata que a escolha de um \emph{hardware} de ponta para a obtenção da mesma disponibilidade dos dados. Diversas empresas que trabalham com quantidades massivas de dados e necessitam de processamento rápido e tolerante a falhas utilizam o \emph{Hadoop}, entre elas:

\begin{itemize}
\item Amazon;
\item Adobe;
\item eBay;
\item Facebook;
\item Google;
\item IBM;
\item Last.fm;
\item LinkedIn;
\item Spotify;
\item Twitter;
\item Yahoo!.
\end{itemize}


\section{Caracterização do projeto de código livre} 
Pesquise sobre o projeto para responder as seguintes questões.


\subsection{Desenvolvimento}


\begin{itemize}
\item Existe um local dedicado para o desenvolvimento?

Não foram encontrados indícios de que a ferramenta possui um local dedicado para o desenvolvimento. Na verdade, a existência de um corpo de desenvolvedores de diferentes fusos horários permite a conclusão de que o desenvolvimento é feito mundo afora, tendo a rede como plataforma de comunicação entre os voluntários que trabalham na ferramenta.

\item É possível extrair o atual código fonte a partir de um repositório público de código fonte?

No site da ferramenta é possível fazer o \emph{download} do código fonte.

\item Quão grande é o código?

O código descompactado possui cerca de 82mb e é dividido em vários módulos. Trata-se de uma ferramenta bastante complexa, portanto o código é bastante extenso.

\item Quais são as principais linguagens de programação?

A maior parte da ferramenta é escrita em Java, embora uma parcela considerável esteja sendo reescrita em C e C++ por motivos de velocidade e segurança.

\item A utilização do pacote depende de algum outro software proprietário ou de código fonte aberto?

Sim. Para que o \emph{Hadoop} possa ser compilado, os seguintes requerimentos devem ser cumpridos (considerando o caso em que o sistema utilizado será UNIX):
\begin{itemize}
\item JDK 1.6 ou superior;
\item Maven 3.0 ou superior;
\item ProtocolBuffer 2.5.0;
\item CMake 2.6 ou superior;
\item Findbugs 1.3.9 (opcional, apenas se o usuário quiser utilizar tal ferramenta);
\item Zlib devel;
\item openssl devel.
\end{itemize}

\item É possível calcular o número de \textit{downloads} ou usuários de uma versão em particular?

A informação para este tópico não foi encontrada nos sites da ferramenta.
\end{itemize}

\subsection{Licença Software Livre}


\begin{itemize}
\item Quem são os patrocinadores que contribuem para a sustentabilidade do projeto?

O projeto possui diferentes níveis de patrocínio, que são classificados como Platina, Ouro, Prata, Bronze e Apoio de Infraestrutura. Entre os patrocinadores Platina, encontram-se empresas como \emph{Facebook}, \emph{Microsoft} e \emph{Google}. Já a \emph{IBM} e \emph{HP} encontram-se entre os patrocinadores Ouro. \emph{Pivotal} e \emph{Huawei} são alguns dos patrocinadores Prata do projeto, enquanto \emph{Twitter} e \emph{Samsung} são exemplos de patrocinadores Bronze. Por fim, algumas empresas e universidades são apoiadoras de infraestrutura do projeto. Um exemplo é a empresa de segurança \emph{Symantec}.

\item Quem detém os direitos autorais do código?

Os direitos autorais do \emph{Hadoop} pertencem à ASF (\emph{Apache Software Foundation}).

\item O projeto está sob qual tipo de licença de código aberto?

O projeto se encontra sob a licença \emph{Apache}, versão 2.0.

\item Por que os responsáveis pelo projeto escolheram a licença de código aberto?

O tipo de licença escolhida faz parte da cultura da fundação, visto que todos os seus projetos respeitam a mesma política, na qual todas as empresas que utilizam o código e fazem modificações no mesmo devem utilizar o selo \emph{Powered By}. Em outras palavras, apenas as \emph{releases} lançadas pela \emph{Apache} podem receber o nome \emph{Apache Hadoop}, ou serem declaradas como distribuições dela. Além disso, pode-se supor que, pelo fato da ferramenta ser utilizada por grandes empresas de vários ramos, é muito interessante em termos de \emph{marketing} que o selo \emph{Powered By} seja utilizado pelas mesmas. Além disso, a abertura da licença permite que desenvolvedores voluntários de diversas empresas e com diversas especialidades contribuam com o projeto, tornando-o cada vez mais completo e robusto.
\end{itemize}

\subsection{Governança}


\begin{itemize}
\item Existem quantos desenvolvedores alocados para o projeto?

O projeto possui vários de desenvolvedores alocados no momento. São todos voluntários, divididos entre \emph{Project Management Committee} (PMC) e \emph{committers}, sendo que um não exclui o outro (um membro pode fazer parte dos dois grupos), os quais trabalham em diferentes empresas como \emph{Yahoo!}, \emph{Facebook}, \emph{Hortonworks}, \emph{Cloudera}, \emph{Microsoft}, \emph{VMware}, \emph{Twitter}, \emph{IBM} e \emph{Intel}.

\item Quantos \textit{committers}, também conhecidos por desenvolvedores que podem realizar mudanças propostas, o projeto possui?

Atualmente, o projeto possui 81 \emph{committers} ativos. Entretanto, as listas de \emph{e-mails} possuem centenas de usuários que contribuem com dúvidas, experiências e sugestões.

\item O que você pode dizer sobre o modelo de governança de código fonte aberto?

O projeto apresenta um modelo bastante interessante, visto que membros de diversas empresas podem contribuir para a sua construção, o que torna o código bastante completo, otimizado para a realidade dos seus usuários, seguro e menos exposto a falhas de maneira geral, visto que o código-fonte é aberto e mesmo os usuários que não são \emph{committers} podem relatar experiências nas listas de \emph{e-mails}, o que enriquece todo o processo de desenvolvimento. Por outro lado, a adoção da licença \emph{Apache} é interessante em termos de \emph{marketing}, conforme mencionado, além de impedir que o projeto seja associado a uma única empresa. Por fim, nada impede que desenvolvedores autônomos façam modificações no código, desde que a licença seja respeitada. Em outras palavras, todos podem utilizar a ferramenta e modificá-la de acordo com as suas necessidades, não apenas grandes empresas, embora sejam estas que contribuem com \emph{committers} e com patrocínio financeiro.
\end{itemize}

\subsection{Manutenção}


\begin{itemize}
\item Gerenciamento de \textit{releases}: Qual o número e frequência de \textit{releases}?

O projeto possui 67 \emph{releases} entre 4 de setembro de 2007 e 19 de novembro de 2014. Em outros termos, uma nova \emph{release} é lançada a cada um mês e 8 dias, em média. O número citado contabiliza apenas as \emph{releases} oficiais, mas diversas empresas possuem versões modificadas da ferramenta.

\item Comunicação: Existe um canal de comunicação útil e ativo para a comunidade/suporte ao usuário?

Além do site oficial, que contém todas as releases com as respectivas mudanças/correções, existem listas de \emph{e-mails} nas quais os usuários podem postar dúvidas e compartilhar experiências. Trata-se de uma abordagem interessante, visto que gera acúmulo de conhecimento e permite a resposta de maneira mais rápida e completa, visto que as postagens poderão ser vistas por todos os desenvolvedores e pelos demais usuários da ferramenta.

\item Existe um \textit{bugtracker} (rastreamento de bugs) com uma lista de bugs corrigidos/pendentes de correção?

Sim. O projeto possui 4 \emph{Issue Trackings}, sendo um para o projeto de forma geral, e outros 3 para os módulos que compõem a ferramenta.

\item Existe um plano de metas para planos futuros? Existe evidência que o plano de metas foi utilizado no passado?

Dentro de cada \emph{Issue Tracking} existe um \emph{Road Map} com problemas a serem resolvidos em cada \emph{release} da ferramenta. Pode-se perceber que os \emph{Road Maps} são utilizados de forma efetiva pelos desenvolvedores, ou seja, há evidências de que a abordagem já pautou o lançamento de novas versões da ferramenta.

\item Existe consultoria comercial, treinamento ou consulta disponível para o projeto? A partir de múltiplos prestadores de serviços?  

No site oficial da ferramenta não existem referências, mas a partir de pesquisas por consultoria na área, existem diversas empresas que prestam serviços na área de \emph{Big Data} e que trabalham com \emph{Hadoop}. São alguns exemplos:
\begin{itemize}
\item Inquidia;
\item Altoros;
\item Maestro Technologies;
\item Business Data Miners;
\item WinterCorp.
\end{itemize}       
\end{itemize}
        

\par\vspace{\baselineskip}

%----------------------------------------------------------------------------------------

\end{document}
